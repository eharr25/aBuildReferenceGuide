
\documentclass{article}
\usepackage[utf8]{inputenc}
\usepackage{relsize}
\usepackage{float}
\usepackage{longtable}
\usepackage[dvipsnames]{xcolor}

\title{So You Want to do Materials Research:\\[0.02em]\smaller{}a
guide to aBuild and the skills you need to use it}
\author{Lydia Harris and Eli Harris}

\begin{document}
\maketitle
\begin{abstract}
 The python module aBuild is meant to automate the process of
 building a mtp model for a paticular system. The documents presented
 in this folder/book are meant to be a reference guide to bash, git, aBuild, etc.
\end{abstract}

\section{Getting Started}
To do computational research you will need to learn what the command
line is and some bash commands--see the Bash Commands section. Go
learn your commands it will make your life easier!!\\
Now that you have learned the basics of the commandline environment it
is time to utilize those skills to log into the supercomuter. If you
have not yet made an account, go to marylou.byu.edu click on request
and account. Your faculty mentor (most likely Brother Nelson) will
need to approve your account. Once you have an account...
If you are using mac or linux simply open the terminal and use the ssh
command (see bash commands).  If you have a windows machine you have
several options you can either use the ubuntu app, windows power
shell, widows terminal, ,.... etc.

\subsection{Make directories}
Once logged into the supercomputer you will need to build the
following directories (See Bash Commands):\\
\indent 1. \verb|\home\|codes\verb|\|aBuild\\
\indent 2. \verb|\home\|bin\\
\indent 3. \verb|\home\|environments\\
\indent 2. \verb|\home\|Species-system (i.e. AgPt) 

\subsection{Download aBuild}
Now it is time to download aBuild. The python module aBuild is meant
to automate the process of building a mtp model for a paticular
system. The module is continueuously undergoing improvements. To have
version contronal we use git. If you are unfamiliar with git go read
the git tutorial section. Git is an extreemly helpful tool and used
across many different organizations. Once you become familiar with
git, fork/download aBuild from 
\subsection{Virtual Environment/bash\_script}
Description of how to create a virtual environment and the need
changes to the \verb|.bash_profile|.

\subsection{Copy necessary Vasp information??}

\section{Run aBuild: Training Process}                      
Now that you have your account all set up, you are ready to start
training the model to your specific system. The python module aBuild
is designed to automate the train of the model process. This section
will guild you through the different commands needed to train the model.\\

The commands start with python ``builder.py **YMAL**" and then the
various tags are used to step through the training process. The python
command is only used for the tags denoted by the `-'.

\begin{center}
  \begin{longtable}{||p{4cm}|p{7cm}||} %longtable so it wraps %
    % pages   
    \caption{aBuild steps and description}
    \label{bashcommands}
    \\ \hline
    \textbf{Step} & \textbf{Description}\\ \hline \hline
    \endhead
    % make the previous page footer
    \hline
    \multicolumn{2}{||c||}
    {\tablename\ \thetable\ -- \textit{Continued on next
        page}} \\ \hline
    \endfoot
    %make the last footer
    \hline
    \endlastfoot
    %contents of the table

    -enum & Looks for possible crystal structures to
            enumerate through \textcolor{red}{--run interactively}\\
    -write -rgk & Build the training set folder and run
                  the getKPoints script \textcolor{red}{--run interactively}\\
    -setup\_train & Pull data from VASP folders, Builds
                    train.cfg and pot.mtp \textcolor{red}{--run interactively} \\
    qsub* jobscript\_train.sh & mlp train: needs train.cfg, pot.mtp;
                                generates: Potential.mtp \textcolor{red}{--job
                    submission, parallel, 10-20 cores, 6-12 hrs}\\
    -setup\_relax & Builds to-relax.cfg and relax.ini; runs calc-grade
                    \textcolor{red}{--run interactively}\\
    qusb* jobscript\_relax.sh & mlp relax: needs to-relax.cfg, pot.mtp, and
                                relax.ini; generates: relaxed.cfg,
                                unrelaxed.cfg, and candidates.cfg \textcolor{red}{--job
                    submission, parallel, 10-30 cores, 6-30 hrs}\\
    -setup\_select\_add & Concatenate all of the candidate.cfg\_\# into
                          one candidate.cfg and remove all of the
                          other ones,Concatenate all of the
                          selection.log\_* files into one
                          file. Concatenate all of the relaxed.cfg\_\#
                          into one file.  This file should get bigger
                          and bigger with each iteration, Build a
                          submission script. candidate.cfg \textcolor{red}{--run interactively}\\
    qsub* jobscript\_select & mlp select-add: generates:
                              new\_training.cfg; need train.cfg,
                              candidate.cfg \textcolor{red}{--job
                              submission, single core, 1-4 hrs}\\
    -add & builds A folders in training set and creates jobscript \textcolor{red}{--run interactively}\\
    qsub* jobscript (inside training set) & runs vasp calculations for
                                            unrelaxed configurations
                                            \textcolor{red}{--array
                                            job, 6-30 hrs}\\ 
    go back to step -setup\_train & Repeat until model is fully
                                    trained, i.e. all structures relax.\\
                              
    
  \end{longtable}
\end{center}
*For Marylou use sbatch instead of qsub\\
**YMAL** is the yml file without the .yml extension.

The training process will take several iteration before the model is
able to relax the atoms.

%%%%%%%%%%%%%%%%%%%%%%%%%%%%%%%%%%%%%%%%%%%%%%%%%%%
%More description of the relax and sequential steps
%%%%%%%%%%%%%%%%%%%%%%%%%%%%%%%%%%%%%%%%%%%%%%%%%%%

\section{Theory}
In this section we discuss what the alogorithms are doing or the
theory behind the computations. The overview picture is to create a
MTP model from Vasp calculations. The Vasp calculations are ab initio
or first principle calculations that us DFT and/or DFT+U depending on
the specific system and if spin must be considered. The mtp then is
trained by an optimzation fitting the coefficients of the basis
functions--see MTP for more details. After the model has been trained
it attempts to relax the atoms to their happen place. On the first
iteration the model will likely be unable to relax an of the
configurations. The model then selects more configurations to add to
the train set. These configurations are then calculated using Vasp and
the cycle continues until the model is able to relax all of the
configurations.

The motivation of training a mtp model comes from the bottleneck of
Vasp calculations. In searching for configurations to create the
convex hull the configuration space for vary concentrations is so vast
that using DFT calculation becomes impracticle due to the amount of
time the calculation take. By using an mtp training model we are able
to explore much more of the configuration space in signicantly less
time. Thus, allowing us to potentially discover new configurations of
a paticular system. For further detail see ref \cite{} 

\subsection{MTP}

\subsection{DFT}

\subsubsection{DFT+U}

\subsection{Optimization Routines}

\subsection{Convex Hull}

\subsection{}

\section{Bash Commands}

First things first, you need to learn to navigate your command
line. Macs have a built in command line (terminal). On Windows, there
are several options:\\
\indent 1. Windows Powershell. See:
https://docs.microsoft.com/en-us/windows-
server/administration/windows-commands/powershell
for instructions on how to configure your pc to use it.\\
\indent 2. The Ubuntu app--simply search for it in your windows store.\\
\indent 3. Windows recently came out with Windows Terminal as a 
central place for all the commandline userfaces.\\ 
\indent 4. Other apps such as:...\\
Bash allows to add input into the computer from a powerfull
commandline. See Table
\ref{bashcommands} for a list of helpful commands. 

\begin{center}
  \begin{longtable}{||p{5.5cm}|p{5.5cm}||} %longtable so it wraps pages
    \caption{Bash commands and what they mean}
    \label{bashcommands}
    %make the main header
    \\ \hline
    \textbf{Command} & \textbf{What it does}\\ \hline \hline
    \endfirsthead
    %make the next page header
    \hline
    \multicolumn{2}{||c||}
    {\tablename\ \thetable\ -- \textit{Continued from previous page}}
    \\ \hline
    \textbf{Command} & \textbf{What it does}\\ \hline \hline
    \endhead
    %make the previous page footer
    \multicolumn{2}{||c||}
    {\tablename\ \thetable\ -- \textit{Continued on next
        page}} \\ \hline
    \endfoot
    %make the last footer
    \hline
    \endlastfoot
    %contents of the table
    \verb|ls| & list contents of current directory
    \\ \hline
    \verb|ls -a| & show hidden files too \\ \hline
    \verb|ls -altr| & see the last changes made to
    the files in a directory \\ \hline
    \verb|mkdir directory| & make a new directory
    \\ \hline
    \verb|cd directory| & change directory \\ \hline
    \verb|cd .. |& go back a directory \\ \hline
    \verb|cd ../..| & go back two directories \\ \hline
    \verb|cd ~ or cd |& go to your home directory \\ \hline
    \verb|pwd| & print working directory \\ \hline
    \verb|~/| & means your root \\ \hline
    \verb|.| & means the current directory \\ \hline
    \verb|cp file/to/copy where/newName| & copy
    a file \\ \hline
    \verb|cp file/to/copy .| & copy a file to current
    directory without changing the name \\ \hline
    \verb|cp directory/* . |& copy all the files in a
    directory \\ \hline
    & to the current directory \\ \hline
    \verb|cp -r directory new/directory| & copy a
    directory recursively \\ \hline
    \verb|rm file/to/remove| & remove a file \\ \hline
    \verb|rmdir directory| & remove a directory \\ \hline
    \verb|rm -rf directory| & blow away a directory
    permanently \\ \hline
    \verb|mv file/to/move where/newName| & moves
    or renames a file \\ \hline
    \verb|man command| & show the manual for a
    command \\ \hline
    \verb|cat file/one file/two| \textgreater \verb| new_file|
    & concatonate two or more
    files into a new file \\ \hline
    \verb|history| & shows a history of your
    commands \\ \hline
    \verb|less file/to/see| & shows one page of a
    file \\ \hline
    & space turns the page q quits \\ \hline
    \verb|head file/to/see| & see the first page
    of a file \\ \hline
    \verb|head -n 8 file/to/see| & see the first 8 lines
    of a file \\ \hline
    \verb|tail file/to/see| & see the last page of a file
    \\ \hline
    \verb|tail -n 10 file/to/see| & see the last 10 lines
    of a file \\ \hline
    \verb|grep keyword file/to/search| & search a file
    for a keyword and print all the lines with
    that
    keyword
    to the
    screen \\ \hline
    \verb|history |\textbar\verb| grep keyword| & search your history
    for a keyword \\ \hline
    \verb|grep keyword file/to/search|& count
    the occurences of lines with a \\
    \textbar \verb| wc -l| & keyword\\ \hline
    \verb|command |\textbar \verb| less| & pipe the output of a command
    to less \\ \hline
    \verb|command |\textgreater\textgreater\verb| file|
    & append the output of a
    command to a file \\ \hline
    \verb|command |\textgreater \verb| file| & writes the output of the
    command to a file \\ \hline
    \verb|!command| & executes the most recent command
    that starts with the letters you typed \\ \hline
    \verb|echo something| & print something to the screen \\\hline
  \end{longtable}
\end{center}

\subsection{Bash Loops}

Here's a table \ref{loops} of basic bash loops and logic, and a
basic example \ref{loopexample} that relates to what we do.

\begin{center}
  \begin{longtable}{||p{4.5cm}|p{6.5cm}||}
    \caption{Loops in bash}
    \label{loops}
    %first header
    \\ \hline
    \textbf{Command} & \textbf{What it does}\\ \hline \hline
    \endfirsthead
    %next page header
    \hline
    \multicolumn{2}{||c||}
    {\tablename\ \thetable\ -- \textit{Continued from previous page}}
    \\ \hline
    \textbf{Command} & \textbf{What it does}\\ \hline \hline
    \endhead
    % footer
    \\ \hline
    \multicolumn{2}{||c||}
    {\tablename\ \thetable\ -- \textit{Continued on next
        page}} \\ \hline
    \endfoot
    %last footer
    \hline
    \endlastfoot
    %contents of table
      \verb|for i in {1..100}| & for 100 iterations \\
      \verb|do|& \\
      \verb|command $i| & do this thing(\verb|$i| \\
      \verb|done| & references the index)\\
      \hline
      \verb|if [ condition ]| & check the condition \\
      \verb|then| & if it's true \\
      \verb|command| & do this \\
      \verb|else| & if it's not \\
      \verb|command| & do this \\
      \verb|fi| & \\
      \hline
      \verb|if [ -e file ] |& check if a file exists \\
  \end{longtable}
\end{center}

\begin{center}
  \begin{longtable}{||p{4.5cm}|p{6.5cm}||}
    \caption{Example of a Bash loop}
    \label{loopexample}
    %first header
    \\ \hline
    \textbf{Command} & \textbf{What it does}\\ \hline \hline
    \endfirsthead
    %next page header
    \hline
    \multicolumn{2}{||c||}
    {\tablename\ \thetable\ -- \textit{Continued from previous page}}
    \\ \hline
    \textbf{Command} & \textbf{What it does}\\ \hline \hline
    \endhead
    %footer
    \multicolumn{2}{||c||}
    {\tablename\ \thetable\ -- \textit{Continued on next
        page}} \\ \hline
    \endfoot
    %last page footer
    \hline
    \endlastfoot
    \verb|for i in {1..100}| & for 100 times \\
    \verb|do| & \\
    \verb|cd E.$i| & enter the directory named
    \verb|E.#|\\
    \verb|if [ -e KPOINTS ]| & if KPOINTS doesn't exist
    \\
    \verb|echo $i| & print the directory number \\
    \verb|getKPoints| & run the getKPoints script \\
    \verb|fi|
    \verb|cd ..| & go back one directory \\
    \verb|done|
  \end{longtable}
\end{center}


\section{Emacs}

Emacs is a text editor that has a bunch of cool shortcuts you can
learn to make editing documents super easy. On a Mac you can download
Aquamacs, which uses the same commands as Emacs but you can click with
your mouse. \ref{emacs} Has a chart of basic Emacs
commands. \ref{aquamacs} has a list of Aquamacs specific things. 

\begin{center}
  \begin{longtable}{||p{4.5cm}|p{6.5cm}||}%longtable so it wraps pages
    \caption{Emacs commands and what they mean}
    \label{emacs}
    %first header
    \\ \hline
    \textbf{Command} & \textbf{What it does}\\ \hline \hline
    \endfirsthead
    %next page header
    \hline
    \multicolumn{2}{||c||}
    {\tablename\ \thetable\ -- \textit{Continued from previous page}}
    \\ \hline
    \textbf{Command} & \textbf{What it does}\\ \hline \hline
    \endhead
    %footer
    \multicolumn{2}{||c||}
    {\tablename\ \thetable\ -- \textit{Continued on next
        page}} \\ \hline
    \endfoot
    %last footer
    \hline
    \endlastfoot
    %contents of table
    \verb|emacs path/to/file| & enter emacs editor for
    existing file or creates new file with that name \\ \hline
    ctrl+x ctrl+c y & save and quit a file \\ \hline
    ctrl+x ctrl+c n & quit without saving \\ \hline
    ctrl+w & cut a line \\ \hline
    ctrl+y & paste a line \\ \hline
    ctrl+k & kills the contents of a line \\ \hline
    ctrl+k ctrl+k & kills a whole line \\ \hline
    ctrl+shift+- & undo \\ \hline
    ctrl+u \verb|3 command| & executes the command 3 times \\ \hline
    ctrl+x ctrl+f & find and open a file (at the bottom of the screen)
  \end{longtable}
\end{center}



%%%% DON'T KNOW WHERE THIS BELONGS IN THIS DOCUMENT BUT I'M GONNA TYPE
%%%% IT UP SO THAT WE HAVE IT AND CAN PLACE IT LATER

How to make and use a virtual environment to run python interactively, table \ref{venv}.

\begin{center}
  \begin{longtable}{||p{6.5cm}|p{4.5cm}||}
    \caption{Build a virtual environment}
    \label{venv}
    %first header
    \\ \hline
    \textbf{Command} & \textbf{What it does}\\ \hline \hline
    \endfirsthead
    %next page header
    \hline
    \multicolumn{2}{||c||}
    {\tablename\ \thetable\ -- \textit{Continued from previous page}}
    \\ \hline
    \textbf{Command} & \textbf{What it does}\\ \hline \hline
    \endhead
    %footer
    \multicolumn{2}{||c||}
    {\tablename\ \thetable\ -- \textit{Continued on next
        page}} \\ \hline
    \endfoot
    %last footer
    \hline
    \endlastfoot
    %contents of table
    \verb|mkdir environments| & make a directory to hold your
    environments \\ \hline
    \verb|cd environments| & go to that directory \\ \hline
    \verb|python -m venv name_of_env| & make the
    environment \\ \hline
    \verb|emacs .bash_profile| & edit your
    \verb|.bash_profile|\\ \hline
    \verb|function workon {| & add these lines to it \\
      \verb|source ~/environments/$1/bin/activate| & \\
      \verb|}| & \\ \hline
    ctrl+x ctrl+c y & save and quit the editor \\ \hline
    \verb|source .bash_profile| & update your
    \verb|.bash_profile|\\ \hline
    \verb|workon name_of_env| & enters your environment
    \\ \hline
    ctrl+d & exits your environment \\ \hline
  \end{longtable}
\end{center}

\section{Basics of Github}

This section is by no means comprehensive. In fact, there's probably a
lot of things missing. If you'd like to actually get good at Github,
you have a lot more work to do. But here's some basics.

The basic idea of Github is that several people can work on the same
code at the same time. It also has version control: if you don't like
the recent changes to your code, you can get rid of them by going back
to an old version of the code. You can have a copy of the code on your
machine to work on, and you can push your changes to the main copy of
the code. Most people, however, don't have their repositories open to
the public to edit. They will have to clear any suggested changes that
you push. But as they make edits, you can pull their copy down to your
machine. Github will warn you if there are any changes you have made
to the code that will be overwritten by copying their changes. It's a
powerful code sharing tool, but takes a little getting used to.

First you need to go make a Github account. Then you can fork (make a
copy of) whatever repository of code that you want to copy and
edit. If you don't want to make edits to the code yourself, you don't
have to make a fork, you can just copy directory from the original
repository each time.
To give your computer (or a remote computer) access to your repositories
on Github, add your computer's public SSH key to Github. How to do
this:
On your command line execute the following command to copy
your public ssh key:
\begin{verbatim}
pbcopy < ~/.ssh/id_rsa.pub
\end{verbatim}
Note: If this didn't work, it means you don't have a public key and will
need to generate one. See the subsection below this for instructions
on how to do that. 

Now you will be able to paste this key into your account on Github. To
do this, go to Settings>SSH and GPG keys>New SSH key. Make sure you
name it something intuitive (e.g. My Mac, Marylou, etc.).

Now you can copy code from Github to your machine. It's
probably a good idea to make a directory to hold all the stuff you're
going to copy with a name you'll recognize later (e.g. \verb|aBuild/|
for aBuild, etc.). Usually these directories are in
\verb|~/codes/|. Now there are several scenarios involving the code
you're copying from Github. Here's a couple and what to do about them:

1. If you just want to make a copy of the code and don't
expect any changes to be made to it (like ever), use the following command:
\begin{verbatim}
git clone git@github.com:user/repository.git
\end{verbatim}
You can do this over and over again, it will just overwrite the code
you copied to your machine. 

2.
a. If you expect to make your own changes to the code, start by making
a fork of the original code on Github. Then add a master copy of your
fork to your machine using the following command:
\begin{verbatim}
git remote add master git@github.com:user/repository.git
\end{verbatim}
b. If you have your own code that you would like to edit and put on
Github for others to see, you will have to make your own repository on
Github, then use the same command as above.

3. If you want to be able to add someone else's updates to the code to
your machine, the convention is to call their repository
\verb|upstream|. You can add read only access to their repository to
your machine with the following command:
\begin{verbatim}
git remote add upstream git@github.com:user/repository.git
\end{verbatim}

Now that you have a copy of the code, here's how to make changes to
the code on your machine, and on Github from your machine. Here's a
list of some basic Git commands \ref{git}. Remember that this is not a
comprehensive list.

\begin{center}
  \begin{longtable}{||p{5.5cm}|p{5.5cm}||}
    \caption{Basic Github commands}
    \label{git}
    %first header
    \\ \hline
    \textbf{Command} & \textbf{What it does}\\ \hline \hline
    \endfirsthead
    %next page header
    \hline
    \multicolumn{2}{||c||}
    {\tablename\ \thetable\ -- \textit{Continued from previous page}}
    \\ \hline
    \textbf{Command} & \textbf{What it does}\\ \hline \hline
    \endhead
    %footer
    \multicolumn{2}{||c||}
    {\tablename\ \thetable\ -- \textit{Continued on next
        page}} \\ \hline
    \endfoot
    %last page footer
    \hline
    \endlastfoot
    \verb|git checkout file/to/update| & make a copy of upstream's
    version of the file on your machine \\ \hline
    \verb|git pull upstream master| & update your master copy
    with upstream's version \\ \hline
    \verb|git status| & tells you the status of each file on your
    machine with Github's copy \\ \hline
    \verb|git add file/to/add| & add your version of the file to the
    commit \\ \hline
    \verb|git commit| & commit the changes
    you added. Will open an editor for you to leave a comment\\ \hline
    \verb|git commit -m "comment"|& commit the changes
    you added and leave a comment \\ \hline
    \verb|git push| & pushes your latest commit to your fork on Git \\ \hline
    \verb|git push master upstream| & push your
    request to add your changes to upstream \\ \hline
    \verb|git log| & See history of most recent commits \\ \hline
  \end{longtable}
\end{center}

Note: Atlassian has a good Git tutorial and reference guide. 

\subsection{SSH key generation}

%%%%% WE NEED TO FIGURE OUT HOW TO DO THIS AGAIN. CAN'T FIND THE
%%%%% WEBPAGE WE RECEIVED OUR INSTRUCTIONS FROM
See https://help.github.com/en/articles/generating-a-new-ssh-key-and-adding-it-to-the-ssh-agent\#generating-a-new-ssh-key

\section{Initialization Process}
To start working there are a series of steps and proceedure that must
be followed. The first step is to obtain an account to the supercomputer--likely
marylou. After setting up and account and logging on, you will need to
make several directories and download aBuild.  
\subsection{aBuild Download}
1. Become familiar with git --see git tutorial

\subsubsection{bin directory}

\section{aBuild tags} %I don't know if this needs to be it's own
% section. Just wanted to make a table of
% dependencies and didn't know where it should
% go.

%%%%% The following two tables might be better suited to be
%%%%% combined. It would create the dependencies for the workflow so
%%%%% you'd know what order to execute all the commands in.

\begin{center}
  \begin{longtable}{||p{3cm}|p{4cm}|p{4cm}||}
    \caption{aBuild tag dependencies}
    \label{aBuilddepend}
    %first header
    \\ \hline
    \textbf{Tag:} & \textbf{Needs:} & \textbf{Generates:}\\ \hline \hline
    \endfirsthead
    %next page header
    \hline
    \multicolumn{3}{||c||}
    {\tablename\ \thetable\ -- \textit{Continued from previous page}}
    \\ \hline
    \textbf{Tag:} & \textbf{Needs:} & \textbf{Generates:}\\\\ \hline \hline
    \endhead
    %footer
    \multicolumn{3}{||c||}
    {\tablename\ \thetable\ -- \textit{Continued on next
        page}} \\ \hline
    \endfoot
    %last page footer
    \hline
    \endlastfoot
    %contents of table
    \verb|-write| & \verb|.yml| file & VASP input files \\ \hline
    \verb|-setup_train| & VASP output files & \verb|train.cfg| \\ 
    & \verb|Potential.mtp| & \verb|pot.mtp| \\ \hline
    \verb|-setup_relax| & %I think something goes here but not
    % sure
    &\verb|to_relax.cfg| \\
    & & \verb|relax.ini| \\ \hline
    \verb|-setep_select_add| & %???
    & \verb|candidate.cfg| \\ \hline
    \verb|-add| & \verb|new_training.cfg| & VASP input files \\ \hline
  \end{longtable}
\end{center}

\begin{center}
  \begin{longtable}{||p{3cm}|p{4cm}|p{4cm}||}
    \caption{mlp command dependencies}
    \label{mlpdepend}
    % first header
    \\ \hline
    \textbf{Command} & \textbf{Needs} & \textbf{Generates}\\ \hline \hline
    \endfirsthead
    %next page header
    \hline
    \multicolumn{3}{||c||}
    {\tablename\ \thetable\ -- \textit{Continued from previous page}}
    \\ \hline
    \textbf{Command} & \textbf{Needs} & \textbf{Generates}\\ \hline \hline
    \endhead
    %footer 
    \multicolumn{3}{||c||}
    {\tablename\ \thetable\ -- \textit{Continued on next
        page}} \\ \hline
    \endfoot
    %last page footer
    \hline
    \endlastfoot
    %contents of table
    \verb|train| & \verb|train.cfg| & \verb|Potential.mtp| \\
    & \verb|pot.mtp| & \\ \hline
    \verb|relax| & \verb|to_relax.cfg| & \verb|relaxed.cfg| \\
    &\verb|relax.ini| & \verb|unrelaxed.cfg| \\
    & \verb|pot.mtp| & \verb|candidates.cfg| \\ \hline
    \verb|select-add| & \verb|train.cfg| & \verb|new_training.cfg| \\
    & \verb|candidate.cfg| & \\ \hline
  \end{longtable}
\end{center}

\bibliographystyle{ieeetr}
\bibliography{refs}

\end{document}


