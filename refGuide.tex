
\documentclass{article}
\usepackage[utf8]{inputenc}
\usepackage{relsize}
\usepackage{float}
\usepackage{longtable}

\title{So You Want to do Materials Research:\\[0.02em]\smaller{}a
guide to aBuild and the skills you need to use it}
\author{Lydia Harris and Eli Harris}

\begin{document}

\maketitle

\section{Bash Commands}

First things first, you need to learn to navigate your command
line. Macs have a built in command line (terminal), but on Windows you will need
to download one first. The Ubuntu app works great. See Table
\ref{bashcommands} for a list of helpful commands. 

\begin{center}
  \begin{longtable}{||p{5.5cm}|p{5.5cm}||}
    \caption{Bash commands and what they mean}
    \label{bashcommands}
    \hline
    \textbf{Command} & \textbf{What it does}\\
    \hline \hline
    \endfirsthead
    \hline
    \multicolumn{2}{||c||}
    {\tablename\ \thetable\ -- \textit{Continued from previous page}} \\
    \hline
    \textbf{Command} & \textbf{What it does}\\
    \hline \hline
    \endhead
    \multicolumn{2}{||c||}
    {\tablename\ \thetable\ -- \textit{Continued on next
        page}} \\
    \hline
    \endfoot
    \hline
    \endlastfoot
    \verb|ls| & list contents of current directory
    \\ \hline
    \verb|ls -a| & show hidden files too \\ \hline
    \verb|ls -altr| & see the last changes made to
    the files in a directory \\ \hline
    \verb|mkdir directory| & make a new directory
    \\ \hline
    \verb|cd directory| & change directory \\ \hline
    \verb|cd .. |& go back a directory \\ \hline
    \verb|cd ../..| & go back two directories \\ \hline
    \verb|cd ~ |& go to your root \\ \hline
    \verb|pwd| & print working directory \\ \hline
    \verb|~/| & means your root \\ \hline
    \verb|.| & means the current directory \\ \hline
    \verb|cp file/to/copy where/newName| & copy
    a file \\ \hline
    \verb|cp file/to/copy .| & copy a file to current
    directory without changing the name \\ \hline
    \verb|cp directory/* . |& copy all the files in a
    directory \\ \hline
    & to the current directory \\ \hline
    \verb|cp -r directory new/directory| & copy a
    directory recursively \\ \hline
    \verb|rm file/to/remove| & remove a file \\ \hline
    \verb|rmdir directory| & remove a directory \\ \hline
    \verb|rm -rf directory| & blow away a directory
    permanently \\ \hline
    \verb|mv file/to/move where/newName| & moves
    or renames a file \\ \hline
    \verb|man command| & show the manual for a
    command \\ \hline
    \verb|cat file/one file/two| \textgreater \verb|new/file|
    & concatonate two or more
    files into a new file \\ \hline
    \verb|history| & shows a history of your
    commands \\ \hline
    \verb|less file/to/see| & shows one page of a
    file \\ \hline
    & space turns the page q quits \\ \hline
    \verb|head file/to/see| & see the first page
    of a file \\ \hline
    \verb|head -n 8 file/to/see| & see the first 8 lines
    of a file \\ \hline
    \verb|tail file/to/see| & see the last page of a file
    \\ \hline
    \verb|tail -n 10 file/to/see| & see the last 10 lines
    of a file \\ \hline
    \verb|grep keyword file/to/search| & search a file
    for a keyword and print all the lines with
    that
    keyword
    to the
    screen \\ \hline
    \verb|history|\textbar\verb|grep keyword| & search your history
    for a keyword \\ \hline
    \verb|grep keyword file/to/see|\textbar\verb|wc -l| & count
    the occurences of lines with a keyword \\ \hline
    \verb|command|\textbar \verb|less| & pipe the output of a command
    to less \\ \hline
    \verb|command|\textgreater\textgreater\verb|file|
    & append the output of a
    command to a file \\ \hline
    \verb|command|\textgreater \verb|file| & writes the output of the
    command to a file \\ \hline
    \verb|!command| & executes the most recent command
    that \\ \hline
    & starts with the letters you typed \\ \hline
    \verb|echo something| & print something to the screen \\\hline
  \end{longtable}
\end{center}

\section{Emacs}

Emacs is a text editor that has a bunch of cool shortcuts you can
learn to make editing documents super easy. On a Mac you can download
Aquamacs, which uses the same commands as Emacs but you can click with
your mouse. \ref{emacsChart} Has a chart of basic Emacs
commands. \ref{AquamacsChart} has a list of Aquamacs specific things. 

\begin{center}
  \begin{longtable}{||p{5.5cm}|p{5.5cm}||}
    \caption{Emacs commands and what they mean}
    \label{emacs}
    \hline
    \textbf{Command} & \textbf{What it does}\\
    \hline \hline
    \endfirsthead
    \hline
    \multicolumn{2}{||c||}
    {\tablename\ \thetable\ -- \textit{Continued from previous page}} \\
    \hline
    \textbf{Command} & \textbf{What it does}\\
    \hline \hline
    \endhead
    \multicolumn{2}{||c||}
    {\tablename\ \thetable\ -- \textit{Continued on next
        page}} \\
    \hline
    \endfoot
    \hline
    \endlastfoot
      \verb|emacs path/to/file| & enter emacs editor for
                                  existing file \\
                       & or creates new file with that name \\
      ctrl+x ctrl+c y & save and quit a file \\
      ctrl+x ctrl+c n & quit without saving \\
      ctrl+w & cut a line \\
      ctrl+y & paste a line \\
      ctrl+k & kills the contents of a line \\
      ctrl+k ctrl+k & kills a whole line \\
      ctrl+shift+- & undo \\
      ctrl+u \verb|3 command| & executes the command 3 times \\
  \end{longtable}
\end{center}

\begin{center}
  \begin{longtable}{||p{5.5cm}|p{5.5cm}||}
    \caption{Extra things in Aquamacs}
    \label{aquamacs}
    \hline
    \textbf{Command} & \textbf{What it does}\\
    \hline \hline
    \endfirsthead
    \hline
    \multicolumn{2}{||c||}
    {\tablename\ \thetable\ -- \textit{Continued from previous page}} \\
    \hline
    \textbf{Command} & \textbf{What it does}\\
    \hline \hline
    \endhead
    \multicolumn{2}{||c||}
    {\tablename\ \thetable\ -- \textit{Continued on next
        page}} \\
    \hline
    \endfoot
    \hline
    \endlastfoot
      ctrl+x ctrl+f & find and open a file (at the \\
                       & bottom of the screen) \\
  \end{longtable}
\end{center}

\section{Bash Loops}

Here's a table \ref{bashloops} of basic bash loops and logic, and a
basic example \ref{loopexample} that relates to what we do.

\begin{center}
  \begin{longtable}{||p{5.5cm}|p{5.5cm}||}
    \caption{Loops in bash}
    \label{loops}
    \hline
    \textbf{Command} & \textbf{What it does}\\
    \hline \hline
    \endfirsthead
    \hline
    \multicolumn{2}{||c||}
    {\tablename\ \thetable\ -- \textit{Continued from previous page}} \\
    \hline
    \textbf{Command} & \textbf{What it does}\\
    \hline \hline
    \endhead
    \multicolumn{2}{||c||}
    {\tablename\ \thetable\ -- \textit{Continued on next
        page}} \\
    \hline
    \endfoot
    \hline
    \endlastfoot
      \verb|for i in {1..100}| & for 100 iterations\\
      \verb|do|
      \verb|command $i| & do this thing \verb|$i| \\
      \verb|done| & references the index \\
      \hline
      \verb|if [ condition ]| & check the condition \\
      \verb|then| & if it's true \\
      \verb|command| & do this \\
      \verb|else| & if it's not \\
      \verb|command| & do this \\
      \verb|fi| & \\
      \hline
      \verb|if [ -e file ] |& check if a file exists \\
  \end{longtable}
\end{center}

\begin{center}
  \begin{longtable}{||p{5.5cm}|p{5.5cm}||}
    \caption{Example of a Bash loop}
    \label{loopexample}
    \hline
    \textbf{Command} & \textbf{What it does}\\
    \hline \hline
    \endfirsthead
    \hline
    \multicolumn{2}{||c||}
    {\tablename\ \thetable\ -- \textit{Continued from previous page}} \\
    \hline
    \textbf{Command} & \textbf{What it does}\\
    \hline \hline
    \endhead
    \multicolumn{2}{||c||}
    {\tablename\ \thetable\ -- \textit{Continued on next
        page}} \\
    \hline
    \endfoot
    \hline
    \endlastfoot
      \verb|for i in {1..100}| & for 100 times \\
      \verb|do|
      \verb|cd E.$i| & enter the directory named
                       \verb|E.#|\\
      \verb|if [ -e KPOINTS ]| & if KPOINTS doesn't exist
      \\
      \verb|echo $i| & print the directory number \\
      \verb|getKPoints| & run the getKPoints script \\
      \verb|fi|
      \verb|cd ..| & go back one directory \\
      \verb|done|
  \end{longtable}
\end{center}


%%%% DON'T KNOW WHERE THIS BELONGS IN THIS DOCUMENT BUT I'M GONNA TYPE
%%%% IT UP SO THAT WE HAVE IT AND CAN PLACE IT LATER

How to make and use a virtual environment to run python interactively, table \ref{venv}.

\begin{center}
  \begin{longtable}{||p{5.5cm}|p{5.5cm}||}
    \caption{Build a virtual environment}
    \label{venv}
    \hline
    \textbf{Command} & \textbf{What it does}\\
    \hline \hline
    \endfirsthead
    \hline
    \multicolumn{2}{||c||}
    {\tablename\ \thetable\ -- \textit{Continued from previous page}} \\
    \hline
    \textbf{Command} & \textbf{What it does}\\
    \hline \hline
    \endhead
    \multicolumn{2}{||c||}
    {\tablename\ \thetable\ -- \textit{Continued on next
        page}} \\
    \hline
    \endfoot
    \hline
    \endlastfoot
      \verb|mkdir environments| & make a directory to \\
                       & hold your environments \\
      \verb|cd environments| & go to that directory \\
      \verb|python -m venv name_of_env| & make the
                                          environment \\
      \verb|emacs .bash_profile| & edit your
                                   \verb|.bash_profile|\\
      \verb|function work on {| & add these lines to it \\
      \verb| source ~/environments/$1/bin/activate| & \\
      \verb|}|
      ctrl+x ctrl+c y & save and quit the editor \\
      \verb|source .bash_profile| & update your
                                    \verb|.bash_profile|\\
      \verb|workon name_of_env| & enters your environment
      \\
      ctrl+d & exits your environment \\
  \end{longtable}
\end{center}

\section{Basics of Github}

This section is by no means comprehensive. In fact, there's probably a
lot of things missing. If you'd like to actually get good at Github,
you have a lot more work to do. But here's some basics.

The basic idea of Github is that several people can work on the same
code at the same time. It also has version control: if you don't like
the recent changes to your code, you can get rid of them by going back
to an old version of the code. You can have a copy of the code on your
machine to work on, and you can push your changes to the main copy of
the code. Most people, however, don't have their repositories open to
the public to edit. They will have to clear any suggested changes that
you push. But as they make edits, you can pull their copy down to your
machine. Github will warn you if there are any changes you have made
to the code that will be overwritten by copying their changes. It's a
powerful code sharing tool, but takes a little getting used to.

First you need to go make a Github account. Then you can fork (make a
copy of) whatever repository of code that you want to copy and
edit. If you don't want to make edits to the code yourself, you don't
have to make a fork, you can just copy directory from the original
repository each time.
To give your computer (or a remote computer) access to your repositories
on Github, add your computer's public SSH key to Github. How to do
this:
On your command line execute the following command to copy
your public ssh key:
\begin{verbatim}
pbcopy < ~/.ssh/id_rsa.pub
\end{verbatim}
Note: If this didn't work, it means you don't have a public key and will
need to generate one. See the subsection below this for instructions
on how to do that. 

Now you will be able to paste this key into your account on Github. To
do this, go to Settings>SSH and GPG keys>New SSH key. Make sure you
name it something intuitive (e.g. My Mac, Marylou, etc.).

Now you can copy code from Github to your machine. It's
probably a good idea to make a directory to hold all the stuff you're
going to copy with a name you'll recognize later (e.g. \verb|aBuild/|
for aBuild, etc.). Usually these directories are in
\verb|~/codes/|. Now there are several scenarios involving the code
you're copying from Github. Here's a couple and what to do about them:

1. If you just want to make a copy of the code and don't
expect any changes to be made to it (like ever), use the following command:
\begin{verbatim}
git clone git@github.com:user/repository.git
\end{verbatim}
You can do this over and over again, it will just overwrite the code
you copied to your machine. 

2.
a. If you expect to make your own changes to the code, start by making
a fork of the original code on Github. Then add a master copy of your
fork to your machine using the following command:
\begin{verbatim}
git remote add master git@github.com:user/repository.git
\end{verbatim}
b. If you have your own code that you would like to edit and put on
Github for others to see, you will have to make your own repository on
Github, then use the same command as above.

3. If you want to be able to add someone else's updates to the code to
your machine, the convention is to call their repository
\verb|upstream|. You can add read only access to their repository to
your machine with the following command:
\begin{verbatim}
git remote add upstream git@github.com:user/repository.git
\end{verbatim}

Now that you have a copy of the code, here's how to make changes to
the code on your machine, and on Github from your machine. Here's a
list of some basic Git commands \ref{git}. Remember that this is not a
comprehensive list.

\begin{center}
  \begin{longtable}{||p{5.5cm}|p{5.5cm}||}
    \caption{Basic Github commands}
    \label{git}
    \hline
    \textbf{Command} & \textbf{What it does}\\
    \hline \hline
    \endfirsthead
    \hline
    \multicolumn{2}{||c||}
    {\tablename\ \thetable\ -- \textit{Continued from previous page}} \\
    \hline
    \textbf{Command} & \textbf{What it does}\\
    \hline \hline
    \endhead
    \multicolumn{2}{||c||}
    {\tablename\ \thetable\ -- \textit{Continued on next
        page}} \\
    \hline
    \endfoot
    \hline
    \endlastfoot
      \verb|git checkout file/to/update| & make a copy of upstream's \\
                       & version of the file on your machine \\
      \verb|git pull upstream master| & update your master copy
                                        with \\
                       & upstream's version \\
      \verb|git status| & tells you the status of each file on \\
                       & your machine with Github's copy \\
      \verb|git add file/to/add| & add your version of the file \\
                       & to the commit \\
      \verb|git commit| & commit the changes
                          you added. \\
                       & Will open an editor for you to \\
                       & leave a comment\\
      \verb|git commit -m "comment"|& commit the changes
                                      you added \\
                       & and leave a comment \\
      \verb|git push| & pushes your latest commit \\
                       & to your fork on Git \\
      \verb|git push master upstream| & push your
                                        request \\
                       & to add your changes to upstream \\
  \end{longtable}
\end{center}

Note: Atlassian has a good Git tutorial and reference guide. 

\subsection{SSH key generation}
%%%%% WE NEED TO FIGURE OUT HOW TO DO THIS AGAIN. CAN'T FIND THE
%%%%% WEBPAGE WE RECEIVED OUR INSTRUCTIONS FROM

\section{Initialization Process}
\subsection{git download}
\subsection{Make directories}
\subsubsection{bin directory}
\section{Training Process}                      

\section{aBuild tags} %I don't know if this needs to be it's own
% section. Just wanted to make a table of
% dependencies and didn't know where it should
% go.

%%%%% The following two tables might be better suited to be
%%%%% combined. It would create the dependencies for the workflow so
%%%%% you'd know what order to execute all the commands in.
\begin{table}
  \begin{center}
    \caption{aBuild tag Dependencies}
    \label{aBuilddepend}
    \begin{tabular}{r|c|l}
      \textbf{Tag:} & \textbf{Needs:} & \textbf{Generates:}\\
      \hline
      \verb|-write| & \verb|.yml| file & VASP input files \\
      \verb|-setup_train| & VASP output files & \verb|train.cfg| \\
                    & \verb|Potential.mtp| & \verb|pot.mtp| \\
      \verb|-setup_relax| & %I think something goes here but not
      % sure
                                      &\verb|to_relax.cfg| \\
                    & & \verb|relax.ini| \\
      \verb|-setep_select_add| & %???
                                      & \verb|candidate.cfg| \\
      \verb|-add| & \verb|new_training.cfg| & VASP input files \\
  \end{longtable}
\end{center}

\begin{table}
  \begin{center}
    \caption{mlp Command Dependencies}
    \label{mlpdepend}
    \begin{tabular}{r|c|l}
      \textbf{Command:} & \textbf{Needs:} & \textbf{Generates:}\\
      \hline
      \verb|train| & \verb|train.cfg| & \verb|Potential.mtp| \\
                        & \verb|pot.mtp| & \\
      \verb|relax| & \verb|to_relax.cfg| & \verb|relaxed.cfg| \\
                        &\verb|relax.ini| & \verb|unrelaxed.cfg| \\
                        & \verb|pot.mtp| & \verb|candidates.cfg| \\
      \verb|select-add| & \verb|train.cfg| & \verb|new_training.cfg| \\
                        & \verb|candidate.cfg| & \\
  \end{longtable}
\end{center}

\end{document}